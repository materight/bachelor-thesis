\chapter{Conclusions and future works}
\label{cha:conclusions_and_future_works}

First, this chapter briefly describes the conclusions this work sought to address along with the research goals and the methods used to reach these goals. The next section discusses potential improvements and future researches based on the insights of this thesis.

\section{Conclusions}
\label{sec:conclusions}

The results presented in the previous chapter supports one of the initial hypotheses presented in Section \ref{sec:research_problem_and_initial_hypoteses}, that claims that is possible to forecast the waiting queue length by the analysis of the customers’ shopping behavior during the previous periods. By defining a predictive model for the customers’ inflow rate into the shop and the dwell times, it is possible to forecast the arrival rates at the checkouts, that give an idea of the traffic levels that the checkouts will have to handle in the near future. Once the arrival and service rates are established,  by applying queueing theory it is possible to determine the evolution of the waiting queues, as well as the average time each customers will have to wait before being served. The store’s management can then take decisions regarding staff allocation based directly on these forecasts, or can rely on the suggestions given by the system that automatically determines the optimal staffing levels based on the future development of customers traffic. This thesis discuss a queue-length-based optimization, that minimize the staff at the counters while maintaining the queue length under a maximum threshold, but the algorithm can be easily adapted to reflect custom internal store’s queue management policies and staff scheduling constraints.

Considering the randomness of the customers’ behavior, the forecasts obtained with this approach are accurate enough to represent a useful tool to support decision making and resource planning. Given that RetailerIN is focused on layout and staffing optimization to increase store’s profitability, this model provide an additional value to the predictive capacity of the Thinkinside’s system and it is easily implementable, since it was designed to cooperate with the already-existing components and data formats. Moreover, the analyses conducted on this thesis are generic enough to be adapted to other contexts besides supermarkets, for example airports and banks, but also call-centers, as long as measurements for the traffic and service levels are available and the setting is characterized by at least one waiting queue with adjustable service capacity.

\section{Future works}
\label{sec:future_works}

As stated previously, the main objectives of this thesis were to develop a predictive model for how the waiting queues will evolve in the near future and to support management in adjusting staffing levels in order to maximize store operations while maintaining an adequate service level. While the accuracy of the queue length forecast could be evaluated, as described in Chapter \ref{cha:results}, it was not possible to verify the effectiveness of the suggested checkouts optimizations due to time constraints. This could be done by implementing the final model into RetailerIN and by testing it with different supermarkets layout. By ensuring that the suggestions provided are followed by the staff and by collecting the performance metrics over a time span of different months, it would be possible to compare the performance of the influenced and uninfluenced system by analyzing the average queues lengths and amount of time the staff spent at the counters. However, this is impractical for several reasons, the first of which is that it is impossible to guarantee that the checkouts configuration suggestions will be always followed, and therefore it is difficult to obtain performance metrics that are  truly representative of the effectiveness of the model. A less significative but simpler evaluation method, without the aforementioned issue, would be to implement a simulation by using realistic arrival and service rates based on the approximations described in this thesis.

Another possible improvement would be to refine the dwell times distributions by excluding the time spent in queue from the dwell times of each session. Moreover, as already described in Section \ref{subsec:arrival_rate_forecast_results_conclusions}, the arrival rate forecast, and therefore the overall accuracy, could be further improved by implementing a more advanced dwell time prediction by considering in real time the movements of every customers in the store.

\clearpage


