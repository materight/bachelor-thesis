\chapter{State of the art}
\label{cha:state_of_art}

This chapter discuss the available related literature. First, different approaches from similar researches are described. All this researches fall into the broader discipline of \emph{Operational Research} (OR) and mainly into two specific sub-fields: \emph{Time Series Forecasting} and \emph{Queueing Theory}. The state of the art techniques for this fields are discussed in the following two chapters.

Terminology used:
\begin{itemize}
  \item \emph{Inflow/outflow rate}: number of customers that enter/exit the supermarket in a given interval.
  \item \emph{Arrival rate}: number of customers that arrive at the checkouts in a given interval.
  \item \emph{Service rate}: maximum number of customers that can be served by each checkout in a given interval.
  \item \emph{Dwell time}: time spent by a customer in the supermarket.
  \item \emph{Basket size}: number of items bought in a shopping session.
\end{itemize}


\section{Literature review}
\label{sec:literature_review}

There are various studies that try to analyze and predict the queue length in different settings, but little was found about staff optimization.

Berman et al. (2004) \cite{berman} developed a system that manages the switching of workers between a "front room", where the checkouts are located, and a "back room", that is the shop. This system process real-time data about the count of customers either in front and back rooms, with the goal of minimizing the customer's waiting time. To accomplish this performance objective, they first defined a \emph{waiting time threshold} and required it to not be exceed. Second, they defined a minimum \emph{time-average worker complement threshold} to complete all the back room work. They used a \( M/M/\infty \) queue model to determine the performance of the system with a given number of front room workers. With this model, they were able to suggest in real-time the optimal strategy of switching workers from one room to another, minimizing the average waiting time while satisfying the thresholds constraints.

Aksu H. (2018) \cite{aksu} main goal was to optimize the idle time of the staff operating at the checkout while maintaining the waiting queue length under a predetermined threshold. He proposed a model based on the inflow of customers at the entrance, the number of customers entering the checkout area, the current waiting queue length and number of available checkouts. He divided the supermarket area into a shop and a checkout area and used a video-based system to count the customers moving in each area. He then developed a prediction model to calculate a realistic forecast of customer dwell times. The inward flow into the checkout area was predicted using the inward flow into the shop, since the inward flow into the shop appear with a delay at the checkout area, and this delay is the dwell time. To consider non-standard events, the model was adjusted with real-time measured data, to increase the prediction accuracy. Queueing theory was then used to predict the average waiting time at the checkouts. Checkouts can be opened or closed based on a minimum and maximum acceptable queue length values. To avoid closing and opening too frequently, the system was able to decide in how many cases the defined waiting queue length can be exceeded, e.g. it is acceptable to exceed the target value for a short period of time.

\section{Time series forecasting}
\label{sec:time_series_forecasting}

A \emph{time series} is a collection of observations made sequentially through time. \emph{Time series forecasting} is the use of a model to predict future values based on previous observations.

A time series can be decomposed into three main components:
\begin{itemize}
  \item \emph{Trend} \( T_t \): Linear/nonlinear, increasing/decreasing behavior of the series over time.
  \item \emph{Seasonality} \( S_t \): Repeating patterns or cycles behavior over time.
  \item \emph{Residual} \( R_t \): Variability of the observations not explainable by the model.
\end{itemize}

There are two types of decomposition: \emph{additive}, where the time series would be written as \( y_t = T_t + S_t + R_t \), and \emph{multiplicative}, where \( y_t = T_t \cdot S_t \cdot R_t \).
The additive decomposition is the most appropriate if the magnitude of the seasonal fluctuations, or the variation around the trend-cycle, does not vary with the level of the time series. When the variation in the seasonal pattern, or the variation around the trend-cycle, appears to be proportional to the level of the time series, then a multiplicative decomposition is more appropriate.

Once the main components of the time series have been determined, a predictive model can be defined. There are many different time series forecasting methods available, the most popular are presented in the next sections.



\subsection{Exponential smoothing}
\label{subsec:exponential_smoothing}
In the \emph{Simple exponential smoothing} technique, the forecasted values are based on a weighted average of the previous values, where the most recent observations are given more importance using larger weights \cite{hyndman}.

A more complex implementation is the \emph{Holt-Winters exponential smoothing} approach, which is able to decompose the time series into a level, trend and seasonal component, giving a more precise forecast. However, this method is unable to model more complex series with multiple seasonality patterns, as in a supermarket inflow rate.

Taylor’s \emph{double seasonal exponential smoothing} method \cite{taylor} was developed to forecast time series with two seasonal cycles: a short one that repeats itself many times within a longer one.


\subsection{Artificial Neural Networks}
\label{subsec:artificial_neural_networks}
\emph{Artificial Neural Networks (ANN)} allows complex nonlinear relationships between the target variable and its predictors. With time series data, lagged values of the time series can be used as inputs, in what is called an \emph{Autoregressive Neural Network (AR-NN)}. With seasonal data, it is also useful to also add the last observed values from the same season as inputs \cite{hyndman}. In general, the input can be expressed as:
\[ (y_{t-1}, y_{t-2}, ..., y_{t-p}, y_{t-m}, y_{t-2m}, ..., y_{t-Pm}) \]
We use the notation \( \text{AR-NN}(p, P, k)_m \) to indicate a neural network with \( p \) previous lagged values and \( P \) lagged values from previous seasonal cycles in the input, \( k \) neurons in the hidden layer, and a seasonal cycle of \( m \) intervals.

For forecasting two steps ahead, the result of one-step forecast can be used in the input.

\subsection{ARIMA models}
\label{subsec:arima_models}
\emph{ARIMA (AutoRegressive Integrated Moving Average)} models provide another approach to time series forecasting. Exponential smoothing and ARIMA models are the two most widely used approaches to time series forecasting, and provide complementary approaches to the problem. While exponential smoothing models are based on a description of the trend and seasonality in the data, ARIMA models aim to describe the autocorrelations in the data.\cite{hyndman}

\emph{AutoRegressive} means the model uses the relationship between an observations and some number of lagged previous observations to generate a linear regression model. \emph{Integrated} refers to the use of differencing of raw observations (e.g. subtracting an observation from an observation at the previous time step) in order to make the time series stationary. A stationary time series has mean, variance and autocorrelation constant over time. \emph{Moving Average} means that the model uses the dependency between an observation and the residual error from a moving average model applied to lagged observations.

The parameters of the ARIMA model are defined as follows:
\begin{itemize}
  \item \( p \): Autoregressive order, the number of lagged observations included in the model.
  \item \( d \): Degree of differencing, the number of times that the raw observations are differenced.
  \item \( q \): Moving average order, the size of the moving average window.
\end{itemize}

A simple version of the ARIMA model can be used to forecast only non-seasonal time series, but it can be extended to model seasonal patterns with the Seasonal ARIMA (SARIMA). There are four seasonal parameters that are not part of ARIMA that must be configured:
\begin{itemize}
  \item \( P \): Seasonal autoregressive order.
  \item \( D \): Seasonal degree of differencing.
  \item \( Q \): Seasonal moving average order.
  \item \( m \): The number of time steps for a single seasonal period.
\end{itemize}
A \( \text{SARIMA}(p,0,0)(P,0,0)_m \) model is equivalent to an \( \text{AR-NN}(p,P,0)_m \) model but with a restrictions on the parameters that ensure stationarity.
However, only a single seasonal effect can be modelled with SARIMA. To model more seasonalities the SARIMAX model can be used, since it supports exogenous variables. The seasonal effects can be captured by Fourier terms and used as exogenous variables. This gives a better approximation and allows other exogenous variables to be considered (e.g. weather, holidays, ...) to further improve the accuracy.

\subsection{TBATS models}
\label{subsec:tbats_models}
An alternative approach proposed by De Livera et al. \cite{de_livera} uses a combination of Fourier terms with an exponential smoothing model and a Box-Cox transformation, in a completely automated manner. The main advantage of a TBATS model over a SARIMAX model with Fourier terms is that the seasonality is allowed to change slowly over time, while harmonic regression terms force the seasonal patterns to repeat periodically without changing. However, one drawback of TBATS models, is that they are slow to estimate, since they will consider various alternatives and fit different models.


\section{Queueing theory}
\label{sec:queueing_theory}

\emph{Queueing theory} is the mathematical study of waiting lines, or queues. A queueing model is constructed so that queue lengths and waiting time can be predicted.

Queueing theory is the mathematical study of waiting lines, or queues. A queueing model is constructed so that queue lengths and waiting time can be predicted. There are various types of queueing models. Kendall's notation is the standard system used to describe and classify these models. In 1953, Kendall [7] proposed to describe queueing models using four factors, written as $ A/S/c/K $, where:
\begin{itemize}
  \item $ A $ denotes the time distribution between arrivals to the queue
  \item $ S $ denotes the service time distribution
  \item $ c $ denotes the number of open servers
  \item $ K $ denotes the maximum capacity of the queue. If not specified it is assumed $ K = \infty $
\end{itemize}

The best representation of a supermarket queue is probably given by the $ M/M/c $ model, where the $ M $ denote a Markovian process, meaning that the inter-arrival times and the service times of the $ c $ servers are exponentially distributed. The customers are served in FCFS order (First Come First Served).
The utilization factor $ \rho = \lambda/c\mu $ describes the proportion of total service capacity being used in the system, where:
\begin{itemize}
  \item $ \lambda $ is the average arrival rate
  \item $ \mu $ is the average service rate of a single server
  \item $ c $ is the number of available servers
\end{itemize}

If $ \rho >= 1 $, i.e. the arrival rate exceeds the total service capacity, then the queue will grow indefinitely, but if $ \rho < 1 $ the system is stable and the average queue length can be calculated.

The main issue with this stationary approach in the context of supermarkets is that the arrival rate and the number of available terminals cannot be expressed with probability distributions with constant mean, since they are strongly time-dependent, thus non-stationary. We can denote such time-varying model as $ M(t)/M/c(t) $. Moreover, it could be possible that the utilization factor exceeds 1 for a short period of time (overloading), in which the queue length increases, and after that, when $ \rho < 1 $, the system returns slowly to its equilibrium.

A common approach for dealing time-varying rates is the Stationary Independent Period by Period (SIPP) approach, where the analysis is conducted on small intervals. For each interval, a different $ M/M/c $ model with constant arrival rates and constant number of available servers is created and solved with the stationary approach. However, Green et al. (2001) [8] showed that the commonly used SIPP approach is inaccurate for parameter values corresponding to many real situations, even when the time-dependent variations are small and especially for systems which operate near the critical load. Moreover, this stationary analysis requires the arrival rate to be strictly smaller than the service rate, i.e. $ \rho < 1 $ must hold for every interval, while many real systems can be temporarily overloaded.

Green et al. [9] proposed an easy-to-compute approximation for determining long run average performance measures for multi-server queues with periodic arrival rates, the Pointwise Stationary Approximation (PSA). This approximation was obtained by computing the expectation of the performance measure over the time interval using the stationary formula with the arrival rate that corresponds to each point in time. They showed that this approximation is an upper bound for the expected number of customers in the queue. However again, the PSA approach can only be applied to systems where $ \rho < 1 $ hold, i.e. temporal overloading in some intervals is strictly forbidden [10].

Rider [11] proposed an approximation of the average queue length for a $ M/M/1 $ model with time-varying parameters. He found that the average queue length $ Q(t) $ satisfies the equation:

$ Q(t) = \lambda(t) - \mu(t) [1 - P_0(t)] $

where $ P_0(t) $ is the probability of the server being idle. This equation can be interpreted to mean that the queue will increase with the arrival rate and decrease with the service rate multiplied by an "efficiency factor" $ 1 - P_0(t) $, i.e. the probability that the server is busy at time $ t $. $ P_0(t) $ is not a priori known, thus an approximation must be used.

Stolletz [10] proposed an improvement of the SIPP approach, the Stationary Backlog-Carryover (SBC) approach. This approach was designed for systems with temporal overloading, thus the models of consecutive periods are no longer independent from each other. Contrary to the SIPP approach that applies the $ M/M/c/\infty $ model to each interval, SBC utilizes the $ M/M/c/c $ model, also known as the Erlang’s B model, meaning that a maximum number of  $ c $ customers can be in the queue at any time and any further arrivals to the queue are considered blocked (i.e. lost) and carried over into future periods. This is obtained by defining an artificial arrival rate $ \widetilde{\lambda}(t) $ that consists of both the original arrival rate $ \lambda(t) $ and a backlog rate $ b(t) $ of the previous period, that is the rate of blocked customers leaving the $ M/M/c/c $system at time $ t $.

Once the average queue length has been determined with one of the aforementioned methods, the average waiting time in the queue can be calculated using Little’s Law [12]. This law states that the average number of customers in a stationary system $ L_s $ is equal to the arrival rate $ \lambda $ multiplied by the average time $ W_s $ that a customer spends in the system:

$ L_s = \lambda W_s $

\clearpage