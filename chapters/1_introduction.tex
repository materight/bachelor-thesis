\chapter{Introduction}
\label{cha:introduction}

This chapter comprises three sections. In the first section, the research problem behind this thesis as well as hypotheses are presented. The next section gives an overview of the RetailerIN system. In the last section the general structure of the thesis is described.

\section{Research problem and hypothesis}
\label{sec:research_problem}

This thesis was written as part of an internship at Thinkinside….

\section{RetailerIN system}
\label{sec:retailerin_system}

RetailerIN is an in-store analytics and engagement solution, able to measure and analyse in real-time and with high precision (sub-meter accuracy) how shoppers move and interact with products in a physical store. RetailerIN relies on an indoor location system based on BLE sensors mounted on the store’s physical assets, such as baskets and carts. Since only the assets are tracked, all the data collected is anonymous, not traceable to specific customers. For every asset the RetailerIN system is able to distinguish each customer’s shopping session and construct a “shopping journey” report. This data is then stored and processed in real-time to extract relevant indicators and metrics on the customers’ behavior. These indicators are used to build visual analytics dashboards for store and marketing managers, helping them to optimize the placement of goods, to measure the effectiveness of store layouts and to improve staff management.

\section{Structure of the thesis}
\label{sec:structure_of_the_thesis}

Chapter 3 explores the available literature and introduces different possible approaches to the problem. Chapter 4 describes all the analysis conducted in order to obtain a better understanding of the data as well as identify the most appropriate methods for the solution. In Chapter 5 the final prediction model is introduced and justified. Chapter 6 illustrates the model accuracy and the results obtained by the influenced system. The final chapter provides a conclusion derived from the results and points out some limitations and possible improvements of the system.

\clearpage