\chapter*{Abstract}
\label{cha:abstract}

\addcontentsline{toc}{chapter}{Abstract}

Long waiting queues at the checkouts are one of the major causes of customers’ dissatisfaction in retailing. The time spent waiting for service greatly affects the perceived service quality and therefore can have a significant impact over customers’ loyalty and, eventually, sales. In the context of supermarkets, the only way for store managers to have control over the queues length is by adjusting staffing levels, in an attempt to provide a uniform level of service at all times.

The decision of either increasing or decreasing staffing levels is typically taken on the spot by looking at the current queues’ state. This method is not efficient since it is not responsive enough to sudden changes in traffic levels  and does not take in account the near-term evolution of customers flow. A more efficient way of improving customers’ satisfaction may therefore be to prevent waiting queues before they appear, increasing the number of open terminals according to the predicted inflow of people at the checkouts. However, keeping a checkout open has a cost in terms of manned staff, since it reduces staffing levels for the others store operations, and the benefits of reducing waiting times could be easily overcome by the resulting costs. An optimization which takes in account both cost and customers’ satisfaction is therefore necessary.

The work presented in thesis was developed to be part of RetailerIN by Thinkinside, an advanced in-store analytics and engagement solution, that tracks in real-time the movements of the supermarket’s customers and provides insightful information regarding their shopping behavior, such as identifying the high-traffic areas or the most recurrent shopping journeys. With the support of these performance indicators, retailers are able to optimize the store layout, improve the overall shopping experience and engagement, and receive suggestions on how to adjust staffing levels based on shoppers volume. The main objective of this thesis is to improve the latter feature by developing a predictive model of the queueing behavior in retail stores, with the aim of forecasting near-term arrival rates at the checkouts, supporting the management in a more efficient pre-positioning of the staff to avoid, or at least reduce, waiting queues. The system shall suggest an optimal checkouts configuration, minimizing the staffing costs while maintaining the customers’ waiting times under a maximum threshold. The main hypothesis behind this approach is that it is possible to decrease the total manned time of the staff at the counters without any loss of customer satisfaction, i.e. without increasing the waiting time past an acceptable value.

A predictive model is developed in order to get an accurate forecast of the queue length and to determine the best checkouts configuration. First, a model for forecasting the number of customers entering the store, based on the data of past weeks, is used in order to achieve a multi-step forecast. Next, the measured and predicted inflow rates are combined with the dwell time distributions to get a near-term forecast of the arrival rates at the checkouts. Finally, a predictive model for the expected value of the queue length and waiting time is defined by applying queueing theory. With these components, once a maximum acceptable queue length has been defined, different checkouts configurations can be evaluated and the optimal one can be determined. The methods used to implement these models include: artificial neural networks, ARIMA models, distribution analysis and different queueing theory techniques.

The model thus designed can be easily integrated into RetailerIN, offering store managers near-term forecasts of the traffic levels and suggesting optimal staffing levels in real-time to support operational decision making. It is also possible to adapt the implementation to meet possible store’s specific requirements, in order to adjust to the different layouts and to implement any eventual pre-existing staffing constraints or queue management policy. Finally, all the methodology and the results discussed in this thesis are applicable, with the appropriate adjustments, to other contexts besides supermarkets, as long as they can provide some kind of traceable assets and have at least one waiting queue with adjustable service levels.

\clearpage