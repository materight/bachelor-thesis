\chapter{Results}
\label{cha:results}

The first section of this chapter briefly discuss the implementation of the prototype used for testing and performance evaluation. The following sections present the results obtained by the forecasting techniques described in the previous chapters and justify the choices made for the final model.

\section{Prototype implementation}
\label{sec:checkouts_optimization}
The final prototype was implemented in Python...

\section{Inflow rate forecast}
\label{sec:inflow_rate_forecast}
This section discuss the results obtained by the forecasting methods for the inflow rate presented in Chapter 5.1, in particular:
\begin{itemize}
  \item Persistence model
  \item Naïve model
  \item Artificial Neural Network model
  \item SARIMA model
\end{itemize}

Given the greater amount of available data, the dataset A (see Chapter 4.1) was chosen for the models evaluation. Each model was trained on the first half (from September to November) and tested on the second half (from December to February) of the dataset.

The models prediction accuracy was evaluated by analysing the one-step forecast errors on the test set. A forecast error is the difference between an observed value and its forecast and it can be written as $ e_t = y_t - \hat{y}_t $. Only the store’s opening hours prediction (from 6:00 to 23:00) were considered on the evaluation, since the values outside these periods are equal to zero and therefore not relevant. Different measures were used to summarize the forecast errors:
\begin{itemize}
  \item Mean Absolute Error: $ \text{MAE} = \text{mean}(| e_t |)$
  \item Root Mean Square Error: $ \text{RMSE} = \sqrt{\text{mean}(e_t^2)} $
  \item Mean Absolute Percentage Error: $ \text{MAPE} =  \text{mean}(| 100 e_t / y_t |) $
\end{itemize}

Since the final implementation in the RetailerIN system will work in a real-time environment, the computational performances are another important aspect that must be taken in consideration when choosing the forecasting method. This metric is composed by two measurements: the forecast time and the update time, that respectively indicate the average times spent by the model calculating the forecast value for the next interval and updating the model with new collected data. The forecasting time is the most significant of the two, since the new predictions are calculated with high frequency given the intervals’ short length, while the model updates are executed once every day, preferably when the store is closed. The time spent for the initial training, when the system is started for the first time, is not significant and therefore not considered.

The following table summarizes the results obtained by each model.

\subsection{Conclusions}
\label{subsec:conclusions}
The best performances in terms of forecast accuracy were obtained by the ANN model. However, the naïve model was chosen to be implemented in the final model, since it obtained very similar results and the calculation of a new forecast is faster. Moreover, the architecture of the naïve model is a lot simpler than a neural network, thus it can be implemented directly without needing additional libraries

\section{Arrival rate forecast}
\label{sec:arrival_rate_forecast}
In Chapter 5.2 a forecasting method for the arrival rate was described. To evaluate the accuracy of the aforementioned model’s forecast, a measurement of the actual arrival rate was necessary. Unfortunately this information was not directly available in neither of the two datasets, so an approximation based on the queue behavior was determined, specifically based on the total number of customers in queue, available with one-minute frequency. When the total number of customers in queue increases, a new customer must have joined the queue. Therefore, the positive difference between consecutive intervals values should represent the number of new customers that have just arrived at the checkouts and the arrival rate $ \lambda_t $ can be written as:

$ \lambda_t = Lq_t^{TOT} - Lq_{t-1}^{TOT} + \theta_{t-1} $

Where $ Lq_t^{TOT} $ is the total number of customers in queue at time $ t $ and $ \theta_{t-1} $ is the total number of customers served at time $ t-1 $. The obtained results were then agrregated in 10-minutes intervals to obtain the same frequency as the forecast. The dataset A could not be used for the accuracy evaluation since it does not contain data about the total number of customers in queue. Instead, the evaluation was conducted over the smaller dataset B.

Since this model is part of the time series forecasting field, the same measures for the forecasting errors presented in the previous chapter were used.

\subsection{Conclusions}
\label{subsec:conclusions}

\section{Queue length forecast}
\label{sec:queue_length_forecast}
In Chapter 5.3.2 three approximations of the average queue length were presented, in particular, in the context of the SBC approach:
\begin{itemize}
  \item MAR
  \item A1
  \item A2
\end{itemize}

\subsection{Conclusions}
\label{subsec:conclusions}

\clearpage
